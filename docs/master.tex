\documentclass{article}

\usepackage{amsmath}
\usepackage{amsfonts}
\usepackage{amsthm}
\usepackage{hyperref}

\newtheorem{theorem}{Theorem}[section]
\newtheorem{lemma}{Lemma}[theorem]
\newtheorem{proposition}{Proposition}[theorem]
\newtheorem{corollary}{Corollary}[theorem]

\theoremstyle{definition}
\newtheorem{definition}{Definition}[section]

\theoremstyle{remark}
\newtheorem*{remark}{Remark}

\newcommand{\dig}{\textnormal{dig}}
\newcommand{\sign}{\textnormal{sign}}
\newcommand{\bs}[1]{\boldsymbol{#1}}
\newcommand{\norm}[1]{\left\lVert#1\right\rVert}
\newcommand{\abs}[1]{\left\lvert#1\right\rvert}
\newcommand{\gen}[1]{\langle#1\rangle}
\newcommand{\floor}[1]{\lfloor#1\rfloor}
\newcommand{\ceil}[1]{\lceil#1\rceil}
\renewcommand{\bar}{\overline}
\renewcommand{\vec}[1]{\bm{#1}}      % bold vector style

\newcommand{\rank}{\textnormal{rank}}
\newcommand{\Null}{\textnormal{null}}
\newcommand{\row}{\textnormal{row}}
\newcommand{\Span}{\textnormal{span}}
\newcommand{\col}{\textnormal{col}}
\newcommand{\image}{\textnormal{image}}

\begin{document}

\title{Notes on Mixed Integer Linear Programming}
\author{Martin Sig Nørbjerg}

\maketitle

\section{Introduction}
The term \textit{Mixed Integer Linear Programming (MILP)} arises from the field of mathematical optimization; in general, a \textit{Linear Programming (LP)} problem is an optimization problem of the form:
\begin{align*}
     \min_{x}\; z &= c^T x \\
     \textnormal{s.t.}\; Ax &\leq b\\
      x &\geq 0
\end{align*}
where $x, c \in \mathbb{R}^n, b \in \mathbb{R}^m$ and $A \in \mathbb{R}^{m \times n}$. If the additional constraint that some or all of the entries in $x$ only take on integer values, that is values in $\mathbb{N}$, is imposed then the LP problem transforms into a MILP problem.

\begin{remark}
  The tool Xion, which these notes partly document is a tool for MILP problems. However it should be noted that xion was created purely for educational purposes.
\end{remark}

\begin{definition}
  A MILP is in \textit{standard form} if its of the form:
\begin{align}
     \min_{x}\; z &= c^T x \nonumber\\
     \textnormal{s.t.}\; Ax &= b \label{eq:sf}\\
      x &\geq 0 \nonumber
\end{align}
with $A \in \mathbb{R}^{m \times n}$ and $b \geq 0$.
\end{definition}
\begin{remark}
It is sufficient to consider problems in standard form since, every MILP problem can be converted into a MILP problem in standard form
\end{remark}

\newpage
\subsection{Converting a MILP problem to standard form}
If the objective is $\max_x z = c^T x$, then since $\max_x = -\min_x(-c^Tx)$ we can consider $\overline{c} = -c$. \\ Converting the inequality constraints to equality constraints is less straight forward however consider the following inequality constraint:
\begin{equation}\label{eq:cons}
   a_i^Tx \leq b_i
\end{equation}
Where $a_i$ is the $i$th row of $A$ and $b_i$ is the $i$th entry in $b$. Introducing a slack variable $s_i \geq 0$, allows us to transform the constraint into:
\begin{align*}
    a_i^T x + s_i &= b_i \\
    s_i &\geq 0
\end{align*}
Finally if $b_i < 0$ multiplying both sides of the equation by $-1$ makes sure that the right hand side is positive.
Conversely every constraint of the form:
\begin{equation}
 a_i^T x \geq b_i
\end{equation}
can be written as:
\begin{align*}
    a_i^T x - s_i &= b_i \\
    s_i &\geq 0
\end{align*}

\begin{remark}
Under the hood xion only works with MILP problems in standard form.
\end{remark}

\section{Linear Programming and the Simplex Method}

Let $C$ be a convex set an \textit{extreme point} $p \in C$ is a point, where $x \lambda + (1 - \lambda)y = p$ implies $x = y = p$ for all $\lambda \in (0; 1)$.

Let $P \subseteq \mathbb{R}^n$, then $P$ is called a \textit{polytope} if it is a convex hull of fintely many points in $\mathbb{R}^n$. If there exits a matrix $A \in \mathbb{R}^{n \times m}$ and vector $b \in \mathbb{R}^m$ such that $P = \{x \in \mathbb{R}^n \vert Ax \leq b\}$, then $P$ is called a \textit{polyhedron}.

\begin{theorem}[Representation Theorem]
  Let $A \in \mathbb{R}^{n \times m}$, $b \in \mathbb{R}^{m}$ and $Q := \left\{x \in \mathbb{R}^{n} \vert Ax   \leq b\right\}$ and $P$ be the convex hull of the extreme points in $Q$, and $C = \left\{x \in \mathbb{R}^{n} \vert Ax \leq 0\right\}$. If $\rank(A) = n$ then $Q = P + C := \left\{p + c\vert p \in P, c \in C \right\}$.
\end{theorem}

\begin{definition}
  Let $A \in \mathbb{R}^{m \times n}$ and $b \in \mathbb{R}^{m}_{\geq 0}$ with $\rank(A) = \rank(A, b) = m$ and $n > m$, then a points $\hat{x}$ is called a \textit{basic solution} to \eqref{eq:sf} if $Ax = b$ and the columns of $A$ corresponding to the non-zero components of $\hat{x}$ is linearly independent. Furthermore if $\hat{x} \geq 0$ then $\hat{x}$ is refered to as a \textit{basic feasible solution (BFS)} to \eqref{eq:sf}.
  If more than $n - m$ variables of a basic solution $\hat{x}$ is zero, then $\hat{x}$ is said to \textit{degenerate}.
\end{definition}

\begin{theorem}
  Suppose $A \in \mathbb{R}^{m \times n}$ with $\rank(A) = m$ and $b \in \mathbb{R}^m$, then $\hat{x} \in \{x \in \mathbb{R}^m_{\geq 0} \vert Ax = b \}$ is an extreme point if and only if $\hat{x}$ is a BFS.
\end{theorem}

\end{document}
